\documentclass{article}

\usepackage{cancel}
\usepackage{amsmath}
\usepackage{amssymb}

\begin{document}

	\section*{Exercises for Chapter 1}
	
	\begin{enumerate}
		\item Let $A$ be the set $\{ \alpha , \{ 1, \alpha \}, \{ 3 \}, \{\{ 1, 3 \} \}, 3 \}$. Which of the following
		statements are true or false?
		
		\begin{enumerate}
			\item $\alpha \in A$
			
				True
				
			\item $\{ \alpha \} \in A$
			
				False
				
			\item $\{ 1, \alpha \} \subseteq A$ 
			
				False
				
			\item $\{ 3, \{3\} \} \subseteq A$
			
				True
				
			\item $\{ 1, 3 \} \in A$
			
				True
				
			\item $\{\{1, 3\} \} \subseteq A$
			
				False
				
			\item $\{\{1, \alpha \}\} \subseteq A$
			
				True
				
			\item $\{1, \alpha \} \cancel{ \in } A$
			
				False
				
			\item $\emptyset \subseteq A$
			
				True
		\end{enumerate}
		
		\item Let $B, C, D, E$ be the following sets:
		
			$$B  = \{x | x \text{ a real number}, x^2 < 4 \} ,$$
			$$C = \{x | x \text{ a real number}, 0 \leq x < 2 \},$$
			$$D = \{x | x \in \mathbb{Z}, x^2 < 1 \}. $$
			$$E = \{ 1 \}$$
			
		\begin{enumerate}
		
			\item Which pair of these sets has the propery that neither is contained in the other?
			
				Set $B$ is the same as $-2 < x < 2$. $B$ and $C$ contain common real numbers. 
				$D$ is equal to $-1 < x < 1$. $D$ and $E$ is the pair of sets that neither is contained
				in the other.
				
			\item You are given that $X$ is one of the sets $B, C,  D, E,$ but do not know which one.
				You are also given that $E \subseteq  X$ and $X \subseteq B$. What can you deduce
				about $X$?
				
				Because $E \subseteq X$ we know $1 \in X$. We can deduce from this that $X \cancel{=} D$ as
				$1 \cancel{\in} D$. If $X \subseteq B$ then $X$ could be $B$ or $C$ as ( as $C \subseteq B$) or
				$E$ ( as $E \subseteq B$).
		\end{enumerate}
		
		\item Which of the following arguments are valid? For the valid ones, write down the argument symbolically.
		
		\begin{enumerate}
		
			\item I eat chocolate if I am depressed. I am not depressed. Therefore I am not eating chocolate.
			
				This argument is not valid. The lack of depression says nothing about the eating of chocolate.
				The subject may be eating chocolate when not depressed.
				
			\item I eat chocolate only if I am depressed. I am not depressed. Therefore I am not eating chocolate.
			
				This argument is valid. If $E$ is ``I am eat chocolate'' and $D$ is ``I am depressed'' then 
				``I eat chocolate only if I am depressed'' is represented as $E \Rightarrow D$. If $E \Rightarrow D$
				is true then $\neg D \Rightarrow \neg E$ is also true so the argument is valid.
				
			\item If a movie is worth seeing, then it was not made in England. A movie is worth seeing only if critic
				Ivor Smallbrain reviews it. The movie \emph{The Good, the Bad and the Mathematician} was not
				reviewed by Ivor Smallbrain. Therefore \emph{The Good, the Bad and the Mathematician} was not
				not made in England.
				
				$S$ is `` worth seeing''. $E$ is ``made in England'' and $R$ is ``reviewed by Ivor Smallbrain''.
				
				$$\neg S \Rightarrow \neg E$$
				$$S \Rightarrow R$$
				
				$S \Rightarrow R$ means $\neg R \Rightarrow \neg S$ must be true. So if $\neg R$ is true then $\neg S$
				and so $\neg E$.
		\end{enumerate}
		
		\item $A$ and $B$ are two statements. Which of the following statements about $A$ and $B$ implies one or
			more of the other statements?
			
			\begin{enumerate}
				\item Either $A$ or $B$ is true.
				
				\item $A \Rightarrow B$.
				
				\item $B \Rightarrow A$.
				
				\item $\neg A \Rightarrow B$.
				
				\item $\neg B \Rightarrow A$.
			\end{enumerate}
			
			Statement (a) implies statements (d) and (e).
			
		\item Which of the following statements are true, and which are false?
	
		\begin{enumerate}
		
			\item $n = 3 \text{ only if } n^2 - 2n - 3 = 0$
			
				This is true. $n = 3 \Rightarrow n^2 - 2n - 3 = 0$
				
			\item $n^2 - 2n - 3 = 0 \text{ only if } n = 3$
			
				This is false. $n^2 - 2n - 3 = (n - 3)(n + 1)$ so $n$ could equal $-1$.
				
			\item $n^2 -2n -3 = 0 \text{ then } n =3$
			
				This is false for the same reasons as above.
				
			\item For integers $a$ and $b$, $ab$ is a square only if both $a$ and $b$ are squares.
			
				This is false. 144 is the square of 12. 3 and 48 are integer factors of 144. Neither 3 or 48 are squares.
				
			\item For integers $a$ and $b$, $ab$ is a square if both $a$ and $b$ are squares.
			
				This is true.
				
				$$a = k_{1}^2, b = k_{2}^{2}$$
				$$ab =  k_{1}^{2} k_{2}^{2}$$
				
				$$\text{So } ab = (k_1 k_2)^2$$
				
			
		\end{enumerate}	
		
		\item Write down careful proofs of the following statemens:
		
		\begin{enumerate}
		
			\item $\sqrt{6} - \sqrt{2} > 1$.
		\end{enumerate}
	
	\end{enumerate}
	

	

\end{document}